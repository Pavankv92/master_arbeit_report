The electrode digitalization system purely based on the inexpensive RGB-D camera using machine learning based electrode detection framework, pose-graph based simultaneous localization and mapping, and various cluster processing algorithm was developed and evaluated. The different software modules integrating all the hardware have been implemented using open source software/libraries like miniSam, Open3D, and OpenCV, etc. Static offsets  are determined using hand-eye, eye-in-hand calibrations. The camera calibration is carried out to estimate the intrinsic parameters.  A simple chessboard depth estimation experiment using image point out-projection using raw depth and estimated intrinsic parameters has shown an offset in depth estimation by the camera. For the particular problem at hand, this offset is corrected by comparing the estimated and ground truth points. In the robot-guided case, the average digitalization error 3.7 mm $\pm$ 0.3  for the cap with 63 electrodes, 6.8 mm $\pm$ 2.0 mm  and  5.6 mm $\pm$ 1.2 mm for two caps with 23 electrodes were achieved respectively. For the hand-guided case,  10 mm $\pm$ 2.7 mm for the cap with 63 electrodes, 14.6 mm $\pm$ 3.4 mm, and 19.3 mm$\pm$ 2.3 mm, for caps with 23 electrodes respectively.

Few electrodes are not seen in any of the frames either due to poor camera angle or YOLO failed to detect them. On the other hand, there were false positives and bounding box off position which led to an increased error. Solving these problems could be explored further. The developed algorithm relies on only one single 3D point for each electrode per camera frame for localization. Although this is beneficial in terms of computation time, a small variation in the electrode detection may have an adverse affect on localization. Especially when there are extreme cases where only 3 electrodes are seen per frame. Therefore, it is recommended to include at least few points around the YOLO bounding box center and evaluate its effects on the localization. The Kinect has an inbuilt IMU (inertial measurement unit) and provides a way to record the raw data. Although Microsoft does not provide any readily available tools/algorithms for camera localization using the Kinect IMU data at the time of the thesis, popular approaches like extended Kalman filters (EKF) can be employed to calculate the camera position. Addition of IMU measurement along with ICP and loop closure may lead to better localization. The developed algorithm waits till the end to close the loop and correct the drift. By that time, the accumulation of drift in each sequential ICP registration may be large and beyond correction as in the case of hand held trajectory. Therefore, it is recommended that frequent loop closure is attempted. In this thesis, all the experiments were carried out using a static phantom head which eliminates head movements that occur in real life cases. It would be interesting to apply RGB-D and SLAM based digitalization techniques simulating real life cases.





