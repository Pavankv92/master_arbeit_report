We have developed electrode digitalization system purely based on the RGB-D camera using machine learning based electrode detection framework, pose-graph based simultaneous localization and mapping, and various cluster processing algorithm. Different software modules integrating all the hardware has been implemented using open source software/libraries like miniSam, Open3D and OpenCV, etc. Static offsets  are determined using hand-eye, eye-in-hand calibrations. A necessary evil "camera calibration" is carried out to estimate the intrinsic parameters.  A simple chessboard depth estimation experiment using image point out-projection using raw depth and estimated intrinsic parameters have led to over estimation. For our particular problem, we adjusted over estimation by comparing the estimated and ground truth points. In robot-guided case, average digitalization error 3.7 mm $\pm$ 0.3  for cap with 63 electrodes, 6.8 mm $\pm$ 2.0 mm  and  5.6 mm $\pm$ 1.2 mm for two caps with 23 electrodes were achieved respectively. For hand guided case,  .............................................................................................................................

Few electrodes are not seen in any of the frame either due to poor camera angle or YOLO failed to detect them. On the other hand, we witnessed false positives and bounding box off position, these problems could be explored further. We rely on only one single 3D point for each electrode per camera frame for localization. Although this is beneficial in terms of computation time, small variation in the electrode detection may have adverse affect on localization. Especially when there are extreme cases where only 3 electrodes are seen per frame. Therefore, we recommend the idea of including at least few points around the YOLO bounding box center for localization. Kinect has inbuilt IMU's and provides a way to record the raw data. Although Microsoft doesn't provide any readily available tools/algorithms for camera localization using Kinect IMU data at the time of thesis, popular approaches like extended Kalman filters (EKF) can be employed. Addition of IMU measurement along with ICP and loop closure may lead to better localization. We wait till the end to close the loop and correct the drift. By that time, accumulation of drift in each sequential ICP registration may be large and beyond correction as in the case of hand held trajectory. Therefore, we recommend that frequent loop closure is attempted. Finally, our experiments were carried out using static phantom head which eliminates head movements that occur in real life cases, It would be interesting to apply RGB-D and SLAM based digitalization techniques simulating real life cases.  





