We have demonstrated that electrode digitalization using different EEG caps is possible purely based on the RGB-D camera.

Drift mitigation,
 \begin{enumerate}
 	\item \textit{Dense point cloud:} We rely on only one single 3D point for each electrode per camera frame for localization. Although this is beneficial in terms of computation time, small variation in the electrode detection may have adverse affect on localization. Especially when there are extreme cases where only 3 electrodes are seen per frame. Therefore, we recommend that at least few points around the yolo bounding box center is included for localization.
 	\item \textit{Inertial measurement units (IMU's):} Kinect has inbuilt IMU's and provides a way to record the raw data. Although Microsoft doesn't provide any readily available tools/algorithms for camera localization using Kinect IMU data at the time of thesis, popular approaches like extended Kalman filters (EKF) can be employed. Addition of IMU measurement along with ICP and loop closure may lead to better localization.
 	\item \textit{Frequent loop closure:} We wait till the end to close the loop and correct the drift. By that time, accumulation of drift in each sequential ICP registration may be large and beyond correction as in the case of hand held trajectory. Therefore, we recommend that frequent loop closure is attempted. 
 	
 \end{enumerate}



