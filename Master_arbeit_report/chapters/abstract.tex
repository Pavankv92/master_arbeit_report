\chapter{Abstract}
Electroencephalography (EEG) is used to record electrical activities of the brain using an electrode cap which is placed on the scalp of a specific human. A crucial step of EEG is the accurate estimation of the electrode position on the scalp (digitalization). In this thesis work, An inexpensive RGBD (color+TOF) camera such as Azure Kinect by Microsoft based electrode digitalization system is developed and evaluated on 3 different caps, one with 63 and two with 23 electrodes on them. This system employs machine learning based electrode detection framework, pose-graph based simultaneous localization and mapping, various cluster processing algorithms. for electrode digitalization. These methods are first applied to the data-set created by guiding the camera around phantom head using a robot where the camera motion is precisely controlled. The second part focuses on free hand guided camera. For robot guided case, average digitalization error(post registration) 3.7 mm $\pm$ 0.3  for cap with 63 electrodes, 6.8 mm $\pm$ 2.0 mm  and  5.6 mm $\pm$ 1.2 mm for two caps with 23 electrodes were achieved respectively.

TODO: add hand guided results.

