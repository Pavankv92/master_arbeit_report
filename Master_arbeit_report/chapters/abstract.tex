\chapter{Abstract}
Electroencephalography (EEG) technology is widely used in clinical medicine to monitor the electrical activity of the brain. EEG captures the brain signals with the help of electrodes attached to the scalp. A crucial step in determining the exact source of the brain signal is the accurate estimation of the electrodes position on the scalp known as electrode digitalization. In this thesis work, an inexpensive RGB-D (color+TOF) camera such as Azure Kinect by Microsoft based electrode digitalization system is developed and evaluated on 3 different caps, one with 63 and two with 23 electrodes on them. This system employs a machine learning based electrode detection framework, pose-graph based simultaneous localization and mapping, and various cluster processing algorithms for electrode digitalization. These methods are first applied to the data-set created by guiding the camera around the phantom head using a robot where the camera motion is precisely controlled then on the free hand guided trajectories. For the robot-guided case, average digitalization errors (post registration) of 3.7 mm $\pm$ 0.3  for the cap with 63 electrodes, 6.8 mm $\pm$ 2.0 mm  and  5.6 mm $\pm$ 1.2 mm for two caps with 23 electrodes were achieved respectively. For the hand-guided case, 10 mm $\pm$ 2.7 mm for the cap with 63 electrodes, 14.6 mm $\pm$ 3.4 mm and 19.3 mm$\pm$ 2.3 mm, for caps with 23 electrodes respectively. These methods resulted in lower digitalization accuracy in comparison with widely used electromagnetic based method. 

