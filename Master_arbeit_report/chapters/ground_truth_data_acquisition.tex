\section{Ground truth data acquisition} 
A reflective marker is used to acquire the 3D position of all the electrodes. The marker tip is carefully placed on each electrode and the 3D position is recorded in the tracking camera.  As the Kinect frame is chosen to be the single frame of reference for the entire project, points recorded in the tracking camera are then transferred to the Kinect frame via series of transformations. The  associated systematic errors can be visualized in \cref{fig:error_analysis}. In robot-guided case, these systematic errors consist of two hand-eye calibration errors i.e between tracking camera-robot base and Kinect-robot base. For the hand guided case one eye-in-hand calibration error between Kinect-marker. When out-projected points are compared with ground truth values, there will be an additional out-projection error. 

\begin{figure}[hbt!]
	\centering
	\includegraphics[width=\linewidth]{error_analysis.png}
	\caption{Systematic error in acquiring ground truth data.}
	\label{fig:error_analysis}
\end{figure}

\subsection{Out-projection error determination}  One way to understand the out-projection error is to directly compare the out-projected and ground truth values and by measuring the difference in them. It is already known that chessboard out-projection using raw depth value led to over estimation of the object points depth value. Therefore, we intend to find the parameter $\beta$ that minimizes the error between ground truth and out-projected values with [ raw depth -  $\beta$ ]. This can be addressed in two ways, as shown in  \cref{fig:depth_offset_determination}.  First, at each camera position $x_{1:t} $ both ground truth and out-projected point clouds are known (only those electrodes that are visible from a particular camera pose) and an iterative closest point registration can be performed and output root mean squared error (RMSE) can be used as a metric and averaged over all camera poses. Second, one can stitch the point clouds and find the cluster centers as we did before and compare them to the ground truth values and measure L2 norm. 

\begin{figure}[hbt!]
	\centering
	\includegraphics[width=\linewidth]{depth_offset_determination.png}
	\caption{Depth offset $\beta$ determination process.}
	\label{fig:depth_offset_determination}
\end{figure}
