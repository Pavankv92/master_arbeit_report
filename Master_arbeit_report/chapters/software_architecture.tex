\section{Algorithms for electrode digitalization}

In this section, we provide an overview of the algorithms required for electrode digitalization process which consists of various modules. The individual modules and their input/outputs are shown in \cref{fig:algorithms_for_electrode_digitalization}. The robot guided data set and electrode detection module have been previously implemented and are used in this thesis as a part of the pipeline (orange boxes). The RGB-D camera provides color image and corresponding calibrated depth map. The module \textit{Electrode detection} detects the visible electrodes in each camera frame and looks up the corresponding depth value for each electrode. These two values are combined to form 3D point clouds of electrodes in each frame for further processing. The module \textit{Scan registration}  estimates the relative camera movement between current and previous camera frame by performing the ICP registration between two 3D point clouds. For each camera frame, this module saves the current position relative to the very first camera position (time t =0), relative camera motion estimated previously, and 3D clouds corresponding to current frame (more details on the data structure to follow). The \textit{SLAM} module is responsible for pose-graph creation and optimization. Output is sent to the \textit{Cluster processing} module which finds the final 3D position of electrodes expressed in the first camera frame.  

\begin{figure}[hbt!]
	\centering
	\includegraphics[width=\linewidth]{algorithms_for_electrode_digitalization.png}
	\caption{Overview of algorithms for electrode digitalization.}
	\label{fig:algorithms_for_electrode_digitalization}
\end{figure}

Several libraries have been used to implement each module used for electrode digitalization. Therefore, we give a brief introduction to various libraries used in the rest of this section.

\subsection{ROS} Robot operating system (ROS) \cite{ROS} is an open-source software which serves as middle-ware for robotic software development across many platforms. ROS offers a flexible framework of numerous libraries, software tools for communication, motion planning, navigation, localization and mapping and for robust robotic software development. ROS supports many programming languages including C++, Python, Matlab, etc. A modular and distributed software platform along with a vibrant user community all over the world makes ROS very popular among roboticists, students and researchers. In this project, communication between camera, robot and modules are implemented by exploiting ROS features like publishers and subscribers and visualization tool Rviz has been employed effectively.

\subsection{OpenCV} OpenCV \cite{OpenCV} is a open source computer vision and machine learning library. It provides a flexible framework and highly optimized codes to execute computer vision and machine learning tasks. OpenCV can be used with many programming languages like C++, Python, and comes with Matlab interface. It offers more than 2500 highly optimized codes for performing classical computer vision applications, for example, single/stereo camera calibration, 3D reconstruction, depth mapping etc. and state of the art machine learning tasks such as object detection, human face recognition in the video etc. In this thesis, OpenCV is used mainly for camera calibration and chess board pose estimation. 

\subsection{Matlab} Matlab \cite{WhatIsMA85:online} is a popular platform to design, analyze and simulate real world problems using matrix based language. Matlab provides numerous technology packages/apps to solve problems in fields of science and engineering. In this thesis, Matlab is used for camera calibration. 

\subsection{Open3D} 
Open3D \cite{open3d} is an open-source library loaded with set of tools for 3D data processing such as scan matching (ICP), point cloud visualization, RGB-D image based odometry, etc. Open3D supports development using C++ and Python programming languages. In this thesis, ICP scan registration and point cloud visualization is implemented using open3D.   

\subsection{miniSam}
miniSam \cite{miniSam} is an open-source library which provides factor graph based framework to solve non-linear least square optimization problem. While front-end of miniSam supports c++ and python programming languages, back-end is highly optimized for efficient computation. The task of pose-graph creation and optimization has been efficiently used in this thesis work.  

\subsection{Scikit-learn}
Scikit-learn \cite{scikit-learn} is a open-source python based machine learning framework that supports classification, Regression, Clustering, Dimensionality reduction, etc. Scikit-learn offers various algorithms for cluster processing such as K-Means, DBSCAN, etc. Scikit-learn is mainly used for cluster processing.   



