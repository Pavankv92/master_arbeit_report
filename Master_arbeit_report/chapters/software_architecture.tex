\section{Algorithms for electrode digitalization}

In this section, we provide overview of the algorithms required for electrode digitalization process which consists of various modules. The individual modules and their input/outputs area shown in \cref{fig:algorithms_for_electrode_digitalization}. The robot guided data set and electrode detection module are available as a part of the pipeline (orange boxes). RGB-D camera provides color image and corresponding calibrated depth map. The module \textit{Electrode detection} detects the visible electrodes in each camera scan (frame) and looks up the corresponding depth value for each electrode. These two values are combined to form 3D point clouds for further processing. The module \textit{Scan registration}  estimates the relative camera movement between current and previous scans by performing the ICP registration between two 3D point clouds. For each camera scan, this module saves the current position relative to the very first scan position (time t =0), relative camera motion estimated previously, and 3D clouds corresponding to current scan (more details on the data structure to follow). \textit{SLAM} module is responsible for pose-graph creation and optimization. Output is sent to the \textit{Cluster processing} module which finds the final 3D position of electrodes relative the very first camera scan.  

\begin{figure}[hbt!]
	\centering
	\includegraphics[width=\linewidth]{algorithms_for_electrode_digitalization.png}
	\caption{Overview of algorithms for electrode digitalization.}
	\label{fig:algorithms_for_electrode_digitalization}
\end{figure}

Several open-source libraries has been used to implement each module used for electrode digitalization therefore, we give a brief introduction to various libraries used in the rest of this section.

\subsection{ROS} Robot operating system \cite{ROS} is an open-source software which serves as middle-ware for robotic software development across many platforms. ROS offers a flexible framework of numerous libraries, software tools for communication, motion planning, navigation, localization and mapping and for robust robotic software development. ROS supports many programming languages including C++, Python, Matlab, etc. A modular and distributed software platform along with a vibrant user community all over the world makes ROS very popular among roboticists, students and researchers. In this project, communication between camera, robot and modules are implemented by exploiting ROS features like publishers and subscribers and visualization tool Rviz has been employed effectively.

\subsection{Open3D} 
Open3D \cite{open3d} is an open-source library loaded with set of tools for 3D data processing such as scan matching (ICP), point cloud visualization, RGBD image based odometry, etc. Open3D supports development using C++ and Python programming languages. In this thesis, ICP scan registration and point cloud visualization is implemented using open3D.   

\subsection{miniSam}
miniSam \cite{miniSam} is an open-source library which provides factor graph based framework to solve non-linear least square optimization problem. While front-end of miniSam supports c++ and python programming languages, back-end is highly optimized for efficient computation. The task of pose-graph creation and optimization has been efficiently used in this thesis work.  

\subsection{Scikit-learn}
Scikit-learn \cite{scikit-learn} is a open-source python based machine learning framework that supports classification, Regression, Clustering, Dimensionality reduction, etc. Scikit-learn offers various algorithms for cluster processing such as K-Means, DBSCAN, etc. Scikit-learn is mainly used for cluster processing.   



