Electroencephalography(EEG) is an electrophysiological monitoring method to record the electrical activity of the brain. EEG measures voltage fluctuations resulting from ionic current within the neurons of the brain with the help of electrodes attached to the scalp \cite{EEG}. Due to its non-invasive nature, economy, and ease of use, EEG technology is now widely used in clinical medicine to diagnose epilepsy, coma, brain death, etc.\cite{ASingleCameraPhotogrammetry}. In order to determine the exact source of the brain signal, accurate 3D position of the electodes has to be known. Currently available EEG electrode localization methods are broadly classified as (1) manual methods \cite{binnie1982practical} \cite{de1991practical}, (2) electromagnetic digitalization \cite{khosla1999spatial} \cite{le1998rapid}, (3) ultrasound digitalization \cite{steddin1995new}, (4) magnetic resonance imaging (MRI) assisted methods \cite{koessler2011eeg}, (5) photogrammetric methods \cite{bauer2000measurement} \cite{russell2005geodesic} \cite{koessler2007spatial} \cite{baysal2010single}. The recent developments in the areas of high precision time of flight (TOF) depth cameras have led to the possibility of combing High resolution industrial camera and a TOF depth camera for the EEG electrode digitalization. One such system is used in \cite{chen2019spatial} which captures images from five different perspectives (five fixed camera positions) color (RGB) and depth information is fused together to localize the EEG electrodes.


Variety of ododmetry based positioning techniques are available in the field of robotics. The techniques which relies on the 3D point clouds employ scan registration algorithms which finds the best transformation that aligns 2 subsequent point clouds (source and target). Especially in the era of autonomous driving, these algorithms serves as the basis for localizing the car and simultaneously build the map of the environment also to estimate the car's trajectory by fusing the data from variety of sensor suites (sensor fusion). The main scan registration technique is iterative closest point (ICP) algorithm introduced in \cite{besl1992method} and many varients have been proposed over the years and selecting the appropriate parameters and configuration differs based on the problem. While using ICP, similar to any odometry systems, inherent accumulation of error at each scan registration leads to the drift in the trajectory estimation. Simultaneous localization and mapping (SLAM) technique is usually the go to solution in order to minimize the drift over time and to obtain the best estimate of the trajectory. SLAM in its probalistic form can be addressed via graph-based formulation by constructing a graph whose nodes represent the robot pose and the edge between 2 nodes act as a noisy odometry constraints and loop closure contraints can be added if the robot revisits the previously visited place. Solution to such a graph based SLAM invloves solving a large error minimization problem \cite{grisetti2010tutorial}.     


In this thesis work an inexpensive RGBD sensor such as Microsoft Azure Kinect is hand guided around the head with EEG cap in an specific trajectory while performing ICP registration between the scans for camera localization. 3D point clouds required to perform the registration is available as a previous step of the pipeline. A pose-graph is created on the fly whose nodes represent each camera pose along the trajectory. Output of the ICP scan registration is added as a edge connecting 2 subsequent nodes which serves as odmoetry constraints between 2 subsequent camera poses. Each camera pose along with the associated point cloud is added to the map which is being built simultaneously. Pose-graph is solved each time camera revisits the previously visited places and map will be updated with the corrected poses. The objective of this thesis work is to accuaratly estimate the 3D position of the EEG electrodes by simulataneously localizing the camera and building an EEG electrode map based on scan registration and the pose-graph formulation.


