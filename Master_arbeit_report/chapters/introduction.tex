Electroencephalography(EEG) is an electrophysiological monitoring method to record the electrical activity of the brain. EEG measures voltage fluctuations resulting from ionic current within the neurons of the brain with the help of electrodes attached to the scalp \cite{EEG}. Due to its non-invasive nature, economy, and ease of use, EEG technology is now widely used in clinical medicine to diagnose epilepsy, coma, brain death, etc.\cite{ASingleCameraPhotogrammetry}. In order to determine the exact source of the brain signal, accurate 3D position of the electrodes has to be known. Currently available EEG electrode digitalization methods are broadly classified as (1) manual methods \cite{binnie1982practical} \cite{de1991practical}, (2) electromagnetic digitalization \cite{khosla1999spatial} \cite{le1998rapid}, (3) ultrasound digitalization \cite{steddin1995new}, (4) magnetic resonance imaging (MRI) assisted methods \cite{koessler2011eeg}, (5) photogrammetric methods \cite{bauer2000measurement} \cite{russell2005geodesic} \cite{koessler2007spatial} \cite{baysal2010single}. The recent developments in the areas of high precision time of flight (TOF) depth cameras have led to the possibility of combining high resolution industrial camera and a TOF depth camera for the EEG electrode digitalization. One such system is used in \cite{chen2019spatial} which captures images from five different perspectives (five fixed camera positions) color (RGB) and depth information is fused together to localize the EEG electrodes. The \cref{tab:state_of_the_art} gives an overview of typical digitalization methods.

\begin{table}
	\begin{adjustbox}{width=\columnwidth,center}
	\centering
	\begin{tabular}{|c|c|c|c|c|c|c|}
		\hline
		Method & Principal & Equip size & Time & Accuracy & Reliability & Typical Ref.\\ 
		\hline
		Manual measurement  & Coordinate measuring, calipers  & Small & very slow & 0.4mm & Mid & De Munck et al. (1991) \\
		\hline
		Camera matrix  & Stereo vision  & Large & real-time(<0.1s) & 1.27mm & Bad & De Koessler et al. (2007) \\
		\hline
		Positioning tool  & Electro magnetic digitizer  & Small & 5 min & 2-8 mm & Mid & Datal et al. (2014) \\
		\hline
		Photogrammetry  & Structure from motion  & Small & Slow (5-10) min & 0.8 mm & Mid & Clausner et al. (2017) \\
		\hline
		Laser scanner  & Laser  & Small & Slow & 0.05-0.2 mm & Good & Jeon et al. (2018) \\
		\hline
		Color + depth  & color+TOF  & Small & Real time & 0.3-3.3 mm & Good & Chen S et al. (2019) \\
		\hline
	\end{tabular}
	\end{adjustbox}
	\caption{Comparison of typical methods from \cite{chen2019spatial}}
	\label{tab:state_of_the_art}
\end{table}

Variety of odometry based positioning techniques are available in the field of robotics. The techniques which relies on the 3D point clouds employ scan registration algorithms which finds the best transformation that aligns 2 subsequent point clouds (source and target). Especially in the era of autonomous driving, these algorithms serves as the basis for localizing the car and simultaneously build the map of the environment also to estimate the car's trajectory by fusing the data from variety of sensor suites (sensor fusion). The main scan registration technique is iterative closest point (ICP) algorithm introduced in \cite{besl1992method}. While using ICP, similar to any odometry systems, inherent accumulation of error at each scan registration leads to the drift in the trajectory estimation. Simultaneous localization and mapping (SLAM) technique is usually the go to solution in order to minimize the drift over time and to obtain the best estimate of the trajectory. SLAM in its probabilistic form can be addressed via pose-graph based formulation by constructing a graph whose nodes represent the robot pose and the edge between 2 nodes act as a noisy odometry constraints and loop closure constraints can be added if the robot revisits previously visited place. Solution to such a graph based SLAM involves solving a large non-linear optimization problem \cite{grisetti2010tutorial}. 

In this thesis work, An inexpensive RGBD (color+TOF) camera such as Azure Kinect by Microsoft based electrode digitalization system is developed and evaluated. This system employs machine learning based electrode detection framework, pose-graph based simultaneous localization and mapping, various cluster processing algorithms for electrode digitalization. These methods are first applied to the data-set created by guiding the camera around phantom head using a robot where the camera motion is precisely controlled. The second part focuses on free hand guided trajectories. The objective of this thesis work is to develop algorithms to accurately estimate 3D position of the EEG electrodes by simultaneously localizing the camera relative the head and building an EEG electrode map based on scan registration and the pose-graph formulation.


This thesis is structured in following way. In chapter 2, the required background is provided for various important topics that are necessary to the thesis. For example, a brief introduction to the camera fundamentals including calibration, pose estimation, etc. are first presented. We then present the core algorithms used in this thesis such as scan registration, fundamentals of SLAM especially in pose-graph formulation. We briefly touch electrode detection and cluster processing. In chapter 3, we discuss the algorithmic workflow for electrode digitalization along with various open source libraries used in this thesis. We then outline the experimental setup and different apparatus used. Furthermore, we will elaborate on robot and hand guided data acquisition systems and various calibration process involved along with the challenges encountered. A detailed summary of pose-graph construction, optimization and loop closure included. We close the section by discussing performance metrics used to evaluate the algorithms.
Chapter 4 is dedicated to the results of various calibrations and electrode digitalization. In chapter 5, we discuss few important aspects of the results and final conclusion and future scope is presented in chapter 6.
